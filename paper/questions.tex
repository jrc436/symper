\section{Research Questions}
Due to previous work diminishing the role of symmetry groups in perception, we sought to design a conclusive experiment that tested what features of symmetry were detectable to humans. Further, we wanted to compare multiple different theories of symmetry perception in order to detect which was the most probable. In order to do this, we would design a task requiring people to differentiate wallpaper groups. Our research questions can be summarized as:

\begin{enumerate}
\item {\textbf{\textit{Can people naively distinguish among the wallpaper groups?}} If the answer is yes, then we can expect that, at the very least, more features than reflection play a role in symmetry perception. Second, if people are very good at telling apart the wallpaper groups, then it's possible that symmetry, as defined by mathematics, does actually play a role in perception, fitting in nicely with a computational cognition paradigm.}
\item {\textbf{\textit{What features of symmetry drive the perception of symmetry?}} While most previous studies assumed it was reflection, specifically bilateral reflection, we are not so sure. Clarke's study claimed it was rotation; however, his experimental setup was not able to isolate different variables. If we are able to show that the best predictor of symmetry detection is rooted in the mathematical definition, it would again fit in nicely with the paradigm. Answering this question requires the comparison of multiple different models for how well they fit people's actual performance.}
\begin{enumerate}[a.]
\item {A reflection-based model. This would be in line with previous studies, and it would suggest that the primary predictor of symmetry detection is the presence or absence of reflection axes in various places. \citet{yanxibook} suggests that reflection can be decomposed into four vectors: two diagonal vectors, and two border vectors. We will use that paradigm when testing this model}
\item {A rotation-based model. This would follow Clarke's study, and assume that rotation is the primary driver of symmetry detection. To interpret this model most generously, we would look at each type of rotation for each group: for instance, if 2-fold rotation is the most important feature, then we could assume that 4-fold and 6-fold groups have it, while 3-fold groups do not.}
\item {A subset-based model. This would follow the computational cognition paradigm we are advocating, where perception uses the full set of features as defined by mathematics. To measure this, we would use the shortest distance as defined by Figure~\ref{graph}, which is a good corollary for the groups' mathematical similarity.}
\end{enumerate}
\end{enumerate}