\section{Prior Work}
Despite the formal proof on the topic being over a hundred years old, little work has been done showing these patterns relevance to human perception. Perhaps one such reason is the belief that faces played a special role in our brain's evolution \citep{ffa}, and that is the only symmetry they contain. There are several good works illustrating interesting features of this bilateral reflection symmetry. \citet{bilateral-qual} explains that qualitative features are more important than quantitative differences in symmetry judgments. \citet{bilateral-color} discusses that having to check color drastically increases the time it takes someone to judge something as symmetric.

However, \citet{bio} criticized these and other studies for relying on shapes that are not biological, which it claims is the origin of symmetry detection. It further makes the argument that bilateral vertical reflection is the only symmetry that has much prevalence in nature. While the examples provided should give some evidence of other types of symmetry in nature, we also seek to counter the claim that humans are not good at detecting other types of symmetry. 

The only study that we know of specifically dealing with wallpaper groups is \citet{clarke}. In that study, Clarke found that subjects did not do very well in a symmetry group sorting task. Further, he found that their strategy seemed to be primarily motivated by rotation. In this paper, we argue that the nature of Clarke's task was difficult for reasons other than perceiving the difference between two groups. Further, we argue that the strategy is not simply based on rotation.
