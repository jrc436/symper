\section{Prior Work}
Despite the age of the mathematical work on the topic, little has been done to connect these patterns to human perception. Symmetry beyond bilateral reflection has received little attention. This is perhaps due to work that suggests faces played a special role in our brain's evolution \citep{ffa}; bilateral reflection is the only symmetry they contain. There are several works covering bilateral reflection symmetry. \citet{bilateral-qual} explain that qualitative features are more important than quantitative differences in symmetry judgments. \citet{bilateral-color} discusses that having to check color increases the time it takes someone to determine whether an image is symmetric. They used this to argue that humans do not use color when making such judgments. 

However, \citet{bio} criticized these and other studies for relying on shapes that are not biological, which they claim is the origin of symmetry detection. Generally, many studies use blocks or other generated images rather than those of naturally occurring objects, such as faces or flowers. The authors further make the argument that bilateral vertical reflection is the only symmetry that has much prevalence in nature. While the examples provided should give some evidence of other types of symmetry in nature, we also seek to counter the claim that humans are not good at detecting other types of symmetry. 

The only study that we know of specifically dealing with wallpaper groups is \citet{clarke}. In that study, Clarke found that subjects did not perform well in a symmetry group sorting task. Further, he found that they mostly used rotation in distinguishing among the groups. In this paper, we argue that the nature of Clarke's task was difficult for reasons other than perceiving the difference between two groups. Further, we argue that the strategy is not simply based on rotation.
