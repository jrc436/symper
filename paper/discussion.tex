\section{Discussion}
There are several results that we obtained that require some explanation to integrate with the literature. \citet{clarke} found that people have fairly high inaccuracy in matching the wallpaper groups together and that they seem to primarily rely on rotation in their strategy. However, Clarke's study relies on grouping images together. When presented with this task, it seems unlikely that most people would immediately choose to use seventeen groups. Indeed, the number of sets participants made varied wildly in Clarke's study, from 2 to 23 \citep{clarke}. This makes it somewhat difficult to analyze the actual strategy participants used. Further, if people are to choose some number of groups less than seventeen, rotation may be a logical method to sort them by, as they can cleanly sort them into six different groups. Importantly, Clarke's task does not isolate perception very well. It allows participants an unlimited amount of time, allowing them to use other sorts of analytical reasoning. Second, due to working memory constraints, it would be difficult for participants to recall precisely what type of image corresponded to each group as they continued to sort. These confounds could explain the difference.

\begin{table}
\centering
\begin{tabular}{|l|cccc|}
\hline
& Estimate & Std. Error & z value & Pr$(<|z|)$  \\ \hline
(Intercept) & 0.95971 &  0.09950 & 9.645 & $<2^{-16}$*** \\ \hline
T1 & -0.24795 &  0.04150 & -5.975 & $2.30^{-09}$*** \\ \hline
T2 & -0.07746 & 0.04026 & -1.924 & 0.0544 \\ \hline
D1 & -0.31416 & 0.04063 & -7.732 & $1.06^{-14}$*** \\ \hline
D2 & 0.10197 & 0.04431 & 2.301 & 0.0214* \\ \hline
2fold & 0.01214 & 0.03540 & 0.343 & 0.7316 \\ \hline
3fold & -0.18056 & 0.04197 & -4.302 & $1.69^{-05}$*** \\ \hline
4fold & 0.32353 & 0.04282 & 7.555 & $4.19^{-14}$*** \\ \hline
6fold & -0.07538 & 0.04484 & -1.711 & 0.0870 \\ \hline
tile & -0.07891 & 0.05556 & -1.420 & 0.1555 \\ \hline
distance & 0.22420 & 0.01872 & 11.978 & $<2^{-16}$*** \\ \hline
\end{tabular}
\label{fixeff}
\caption{Logistic Linear Mixed-Effects model predicting accuracy with by-item and by-subject random intercepts. }
\end{table}

In general, this paper advocates for a model of symmetry/pattern detection that is mathematically driven. We believe the subgroup distance to be one method for capturing the method humans perceive regularities. Importantly, it is not the only one and it may not even be the best one. By using Dijkstra's algorithm, we weighted each edge to be exactly equal. However, that's not necessarily the case. While "p1" and "p2" differ by exactly one symmetry (the 2-fold rotation), "p2" and "pgg" differ by three symmetries. Another possible view is to calculate the distance using the Hamming distance between each pair, rather than the relation graph. However, there are clearly multiple definitions of even the same mathematical model when it is operationalized for perception, which complicates the story.

As the model containing only subgroup distance outperformed the other simple models, we find it likely that subgroup distance is the single most useful feature. However, it should be clear from our results displayed earlier that distance alone does not tell the whole story. Thus, we interpret the additional features in the converged model as having an overly important role in the process of pattern recognition. In other words, human perception might use some quick heuristics involving these symmetries instead of doing a full analysis. 

% \begin{table}
% \centering
% \resizebox{\columnwidth}{!}{%
% \begin{tabular}{|l|cccccccccc|}
% \hline
% (Intercept) & T1 & T2 & D1 & D2 & 2fold & 3fold & 4fold & 6fold & tile & distance  \\ \hline
% T1 & -0.191 &   & & & & & & & & \\ \hline
% T2 & -0.090 & -0.556 &  & & & & & & &  \\ \hline
% D1 & -0.216 & 0.111 & 0.040 & & & & & & & \\ \hline
% D2 & -0.051 & -0.005 & -0.137 & -0.572 & & & & & & \\ \hline
% 2fold & -0.472 & 0.113 & -0.010 & 0.136 & -0.061 & & & & & \\ \hline
% 3fold & -0.127 & 0.148 & -0.090 & 0.088 & 0.039 & -0.068 & & & & \\ \hline
% 4fold & -0.464 & 0.179 & -0.121 & 0.172 & -0.245 & 0.160 & 0.237 & & &  \\ \hline
% 6fold & -0.245 & 0.001 & 0.048 & 0.055 & -0.192 & 0.060 & -0.413 & 0.029 & & \\ \hline
% tile & -0.169 & -0.210 & 0.241 & -0.141 & 0.072 & 0.180 & -0.525 & -0.292 & 0.189 & \\ \hline
% distance & -0.713 & 0.052 & 0.082 & 0.046 & 0.116 & 0.405 & -0.029 & 0.296 & -0.026 & 0.350 \\ \hline
% \end{tabular}%
% }
% \label{corr}
% \caption{Correlation of Fixed Effects (note that this is the empirical correlation rather than the analytical relationship)}
% \end{table}

For instance, why are $T_1$ and $D_1$ significant (the vertical and positive diagonal axis), but not $T_2$ and $D_2$ (the horizontal and negative diagonal axis)? While it seems easy to attribute it to perception, it is important to realize there are substantial correlations among features in the wallpaper groups. In some cases, this should be obvious: if a wallpaper group has 6-fold rotation, it is guaranteed to have both 2-fold rotation and 3-fold rotation by the definition. In the case of the reflection axes, there are substantial correlations (about $0.5$) between $D_1$ and $D_2$ and between $T_1$ and $T_2$. Thus, while it's somewhat difficult to disentangle their independent effects, it seems clear that both $T_1$ and $D_1$ are at least more important, as the models including them instead of $T_2$ and $D_2$ perform better on the earlier discussed metrics. However, $D_2$ is also a significant predictor even considering $D_1$, while $T_2$ is not a significant predictor while considering $T_1$. A possible explanation for this could result in faces, with $D_1$ and $D_2$ representing faces at an angle. As mentioned earlier, there are some biological reasons for believing humans are adept at recognizing faces \cite{ffa}.     

The overabundance of the role of reflection symmetry in the literature could be due to the lack of knowledge of researchers of other types of symmetry. Further, it could be related to vast literature on faces, which contains reflection symmetry but lacks all other types. While our results show that people obviously can detect reflection symmetry, it does not tell the whole story.

In the case of rotation, the most significant features were 3-fold and 4-fold. Interestingly, the effect of 4-fold rotation is positive (if both groups both have 4-fold rotation or both don't, accuracy improves), while the effect of 3-fold rotation is negative. This means that groups with 4-fold rotation are not necessarily easy to determine from groups without it, but they are easy to tell apart internally. This is very possibly due to 4-fold rotation's correlation with the other features.

For 3-fold rotation, it is possible that (unlike 4-fold rotation), it plays a more significant role due to its lack of ubiquity. Many objects humans interact with in the modern world contain 4-fold rotation, to the point where perhaps it is so common that it is no longer a useful heuristic. On the other hand, 3-fold rotation is a bit more rare, and thus may have some special significance.

Lastly, the highly significant effect of distance is interesting. The simplest reading of this is that people perceive group to group similarity as more than any individual feature. Computer vision has long been inspired by wallpaper groups \citep{yanxi1,yanxi2}, and this result lends further credence to the idea that they are some fundamental part of human perception.