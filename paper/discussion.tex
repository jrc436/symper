\section{Discussion}
\textbf{This part will need to be revised based on the future work}
Perhaps the most interesting result is that while classic research has focused on reflection, and more recent related research found rotation to be very important, we found neither of those results.

Clarke's study relies on grouping images together. When presented with this task, it seems unlikely that most people would immediately choose to use 17 groups. Indeed, the number of sets participants made varied wildly in Clarke?s study, from 2 to 23 \citep{clarke}. This makes it somewhat difficult to analyze the actual strategy participants used.

Further, if people are to choose some number of groups less than 17, rotation may be a logical method to sort them by. However, Clarke does not seem to consider the role of tile shape, which we found to have a very significant effect.

The overabundance of the role of reflection symmetry in the literature could be due to the lack of knowledge of researchers of other types of symmetry. Further, it could be related to vast literature on faces, which contains reflection symmetry but lacks all other types. While our results show that people obviously can detect reflection symmetry, it is not necessarily what they notice first.

Another interesting note is that while the effect of tile shape is negative, implying (as suspected) that having the same tile shape make the two images more difficult to tell apart, the effect of glide reflection is positive. This could perhaps be somewhat of a lurking variable: for instance, that glide reflection only occurs in patterns that are very distinguishable for other reasons. Alternatively, perhaps glide reflection?s interaction with the other symmetries causes two groups with glide reflection to look very different from each other.

Lastly, the highly significant effect of distance is interesting. The simplest reading of this is that people perceive group to group similarity as more than any individual feature. Computer vision has long been inspired by wallpaper groups \citep{yanxi1}\citep{yanxi2}, and this result lends further credence to the idea that they are some fundamental part of human perception.