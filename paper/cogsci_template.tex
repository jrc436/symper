% 
% Annual Cognitive Science Conference
% Sample LaTeX Paper -- Proceedings Format
% 

% Original : Ashwin Ram (ashwin@cc.gatech.edu)       04/01/1994
% Modified : Johanna Moore (jmoore@cs.pitt.edu)      03/17/1995
% Modified : David Noelle (noelle@ucsd.edu)          03/15/1996
% Modified : Pat Langley (langley@cs.stanford.edu)   01/26/1997
% Latex2e corrections by Ramin Charles Nakisa        01/28/1997 
% Modified : Tina Eliassi-Rad (eliassi@cs.wisc.edu)  01/31/1998
% Modified : Trisha Yannuzzi (trisha@ircs.upenn.edu) 12/28/1999 (in process)
% Modified : Mary Ellen Foster (M.E.Foster@ed.ac.uk) 12/11/2000
% Modified : Ken Forbus                              01/23/2004
% Modified : Eli M. Silk (esilk@pitt.edu)            05/24/2005
% Modified : Niels Taatgen (taatgen@cmu.edu)         10/24/2006
% Modified : David Noelle (dnoelle@ucmerced.edu)     11/19/2014

%% Change "letterpaper" in the following line to "a4paper" if you must.

\documentclass[10pt,letterpaper]{article}

\usepackage{cogsci}
\usepackage{pslatex}
\usepackage{apacite}
\usepackage{graphicx}

\newcommand{\citet}[1]{\citeA{#1}}
\newcommand{\citep}[1]{\cite{#1}}

\title{How to Make a Proceedings Paper Submission}
 
\author{{\large \bf Morton Ann Gernsbacher (MAG@Macc.Wisc.Edu)} \\
  Department of Psychology, 1202 W. Johnson Street \\
  Madison, WI 53706 USA
  \AND {\large \bf Sharon J.~Derry (SDJ@Macc.Wisc.Edu)} \\
  Department of Educational Psychology, 1025 W. Johnson Street \\
  Madison, WI 53706 USA
   \AND {\large \bf Sharon J.~Derry (SDJ@Macc.Wisc.Edu)} \\
    Department of Educational Psychology, 1025 W. Johnson Street \\
    Madison, WI 53706 USA}


\begin{document}

\maketitle


\begin{abstract}
Most literature on symmetry perception has focused on bilateral reflection symmetry with some suggesting that it is the only type of symmetry humans can perceive \citep{bio}. Using image stimuli generated from the mathematically well-defined seventeen wall paper groups, we seek to demonstrate that humans can discriminate various symmetries found in 2D wallpaper patterns \citep{yanxitrends}. Furthermore, we examine which features play an essential role in wallpaper pattern perception. We wanted to test the well-defined features of symmetry along with \textit{subgroup distance}, the shortest path between two groups in the group hierarchy. We found that all groups but one are distinguishable ($p<0.05$) and all are likely distinguishable ($p<0.1$). Further, we found that subgroup distance has a role to play in every interesting model, suggesting it may be a valid model of pattern perception.

\textbf{Keywords:} 
symmetry; visual perception; computational cognition
\end{abstract}


\section{Introduction}
Symmetry has oft been studied as a feature of visual perception and attention (see \citet{review} for one of the many good reviews on the subject). However,  the cognitive science community has focused primarily on bilateral reflection symmetry. The question is: why? This paper seeks to examine why researchers have largely ignored other types of symmetry and to demonstrate that a broader conceptualization of symmetry is useful to studies of human perception and attention.

Symmetry refers to a transformation of an object or image that leaves it with the exact same appearance. There are four primitive types of symmetry: rotation, reflection, glide reflection, and translation. The vernacular term symmetry generally refers to reflection, with the majority of studies focused on bilateral reflection, where the axis is in the center of the object or image.

This paper will focus on \textit{wallpapers}: a type of infinitely repeating symmetric image that is characterized by a specific set of symmetries. First, we will provide the reader with the necessary background to understand this conceptualization, then we will review past work on symmetry, and then we will present a set of findings that show humans are quite adept at detecting other types of symmetry besides bilateral reflection.




\section{Background}
\textit{Symmetry} can be thought of as a transformation on an object that leaves that object exactly the same. This could apply to a variety of objects: for instance, if someone has an infinitely long loop of musical notes, shifting all of the notes forward in time by the length of the loop would not change the music at all. However, this paper focuses on symmetry as it relates to two dimensional images.

\begin{figure}
\centering
\includegraphics[width=0.9\columnwidth]{reflection}
\label{ref}
\caption{On the left, a face, exhibiting bilateral vertical reflection symmetry. On the right, a square, which contains four reflection axes. Reflection axes are marked with blue lines.}
\end{figure}

In two dimensional images, there are only four transformations that are not a simple composition of other transformations. These are \textit{reflection}, \textit{rotation}, \textit{glide reflection}, and \textit{translation}.

Reflection is the most well-known symmetry, with the vernacular for symmetry referring exclusively to it. It refers to when a line can be drawn on an image, and then every point on one side of the line has a corresponding point on the other side. The clearest examples of reflection symmetry in nature are probably the faces or bodies of animals. Reflection is often referred to by the number of reflection \textit{axes}, or lines, that the object contains. A face, for instance, would only contain one. A square, on the other hand, would contain three (See Figure~\ref{ref}).


Rotation is another very common example of symmetry in nature. Rotation refers to an object's ability to rotate around some center without changing. Rotation symmetry is generally referred to by the number of rotation angles that maintain this symmetry. For instance, a hexagon has 6-fold rotation symmetry, while a flower with six pedals would have 6-fold rotation symmetry (See Figure~\ref{rot})

\begin{figure}
\centering
\includegraphics[width=0.9\columnwidth]{rotation}
\label{rot}
\caption{On the left, a flower exhibiting 6-fold rotation symmetry. On the right, a five-pointed star, exhibiting 5-fold rotation symmetry. Rotation centers are marked with a red circle}
\end{figure}

Translation symmetry refers to an object that repeats. For instance, in a brick wall, there are hundreds of bricks aligned in such a way that if you shifted the wall over by the length of exactly one brick, the wall would stay the same. Importantly, for this to make much sense in a mathematical sense, the wall would have to be infinitely long. Another example of translation symmetry would be a fence. If the fence was shifted by the length from one post to the next, the fence would stay the same. An important feature of translation symmetry is the \textit{tile}. The tile refers to the identical feature that is repeating. In the brick wall example, this is an individual brick, while in the fence post example, this is an individual fence post (See Figure~\ref{trans}).

\begin{figure}
\centering
\includegraphics[width=0.9\columnwidth]{translation}
\label{trans}
\caption{On the left, a brick wall. On the right, a white fence. In both of these images, the image would have to repeat to truly exhibit translation symmetry. The tile is outlined in red.}
\end{figure}

The last basic symmetry is glide reflection. While reflection refers to something with a corresponding point directly across from it on the reflection axis, glide reflection refers to something that is reflected and then translated. The clearest examples of glide reflection in nature are footsteps. Each foot could be analyzed in terms of translation symmetry. However, halfway in between each footstep (the tile, in this case), there is a reflected footstep (made by the other foot). Thus, glide reflection is characterized by a "zig-zag" pattern of sorts, where the tile moves and reflects. Importantly, it is impossible to have glide reflection and reflection along the same axis. For instance, if someone walked some path and left footsteps that formed glide reflection, and then someone else walked along filling in a footstep that is a reflection of the original footstep, what would be left is just normal reflection, that happens twice as often as the glide reflection did (See Figure~\ref{glide}).

\begin{figure}
\centering
\includegraphics[width=0.9\columnwidth]{glide}
\label{glide}
\caption{On the left, footsteps exhibiting a glide reflection pattern. On the right, footsteps exhibiting a normal reflection pattern. Note that the two by definition cannot exist along the same axis.}
\end{figure}

In two dimensions, an infinitely large pattern that has translation symmetry is referred to as a \textit{wallpaper}. While all wallpapers have translational symmetry, they can still be divided based on their other symmetries, such as the axes they have reflection along. Interestingly, there are actually only seventeen real combinations of the set of symmetries \citep{wallpaper-proof}. Intuitively, this is because each tile must contain all of the other present symmetries then must also then fit together with an infinite number of tiles of the exact same type. 

\begin{figure}[!ht]
\centering
\includegraphics[width=0.9\columnwidth]{ann_images}
\label{P4GvP4M}
\caption{On the left is P4M, on the right is P4G. In each, one tile is marked with its symmetries. Notice that P4M has reflection axes on its tile border, whereas P4G does not.}
\end{figure}

Every wallpaper has some set of these symmetries, which we will also refer to as features. Each of the seventeen unique combinations are referred to as a wallpaper \textit{group}, with each wallpaper belonging to exactly one group. The symmetric features that define these groups consist of the basic symmetries we've discussed: rotation, reflection, glide reflection, and translation. Each wallpaper group has some set of these features. Among rotation, only 1-fold, 2-fold, 3-fold, 4-fold, and 6-fold tiles are possible. This is simply because other shapes of tiles cannot fit together in an infinite pattern. Table~\ref{sym-tab} gives a basic description of the seventeen groups as a set of their features. While this set has the majority of groups with a unique set of features, those with an equivalent set can be differentiated in other ways, such as what axes in the tile their reflection axes are along. See Figure~\ref{P4GvP4M} for an example. Importantly, any P4M or P4G would have the exact same tile annotations, even if their appearance was different.See Figure~\ref{P4MvP4M} for an example.



\begin{figure}[!ht]
\centering
\includegraphics[width=0.9\columnwidth]{ann_images_same}
\label{P4MvP4M}
\caption{Both images are of the P4M group. Even though their appearance is quite different, they have the same symmetries in every single tile. }
\end{figure}

\begin{table}[!ht]
\centering
\resizebox{\columnwidth}{!}{%
\begin{tabular}{|l|c|c|c|c|c|c|c|c|c|}
\hline
Group & 2-fold & 3-fold & 4-fold & 6-fold & $T_1$ & $T_2$ & $D_1$ & $D_2$ &  tile \\ \hline
P1 & F & F & F & F & None & None & None & None & O \\ \hline
P2 & T & F & F & F & None & None & None & None & O \\ \hline
PM & F & F & F & F & Refl & None & None & None & Re \\ \hline
PG & F & F & F & F & Glide & None & None & None & Re \\ \hline
CM & F & F & F & F & None & None & Refl & None & Rh \\ \hline
PMM & T & F & F & F & Glide & Refl & None & None & Re \\ \hline
PMG & T & F & F & F & Glide & Refl & None & None & Re \\ \hline
PGG & T & F & F & F & Glide & Glide & None & None & Re \\ \hline
CMM & T & F & F & F & None & None & Refl & Refl & Rh \\ \hline
P4 & T & F & T & F & None & None& None & None & S \\ \hline
P4M & T & F & T & F & Refl & Refl & Refl & Refl & S \\ \hline
P4G & T & F & T & F & Glide & Glide & Refl & Refl & S \\ \hline
P3 & F & T & F & F & None & None & None & None & H \\ \hline
P3M1 & F & T & F & F & None & None & Refl& None & H \\ \hline
P31M & F & T & F & F & Refl & Refl & Refl & None & H \\ \hline
P6 & T & T & F & T & Refl & None & None & None & H \\ \hline
P6M & T & T & F & T & Refl & Refl & Refl & Refl & H \\ \hline
\end{tabular}%
}
\label{sym-tab}
\caption{Wallpaper groups represented as their symmetries. (Refl=Reflection, Glide=Glide Reflection, Re=Rectangular, Rh=Rhombic, O=Oblique, S=Square, H=Hexagonal)}
\end{table}

In representing wallpaper groups as a set of symmetries, we can then posit them in relation to each other. In this way, every group is either a subset or a superset of every other group.  If each group is placed with its relation to other groups, they form a graph, see Figure~\ref{graph}. If an arrow points from box $A$ toward box $B$, that means that $B$'s symmetries are subset of $A$'s symmetries.
\section{Prior Work}
Despite the formal proof on the topic being over a hundred years old, little work has been done showing these patterns relevance to human perception. Perhaps one such reason is the belief that faces played a special role in our brain's evolution \citep{ffa}, and reflection is the only symmetry they contain. There are several good works illustrating interesting features of this bilateral reflection symmetry. \citet{bilateral-qual} explains that qualitative features are more important than quantitative differences in symmetry judgments. \citet{bilateral-color} discusses that having to check color drastically increases the time it takes someone to judge something as symmetric, which implies that color is not part of the method humans use to judge something as symmetric.

However, \citet{bio} criticized these and other studies for relying on shapes that are not biological, which it claims is the origin of symmetry detection. It further makes the argument that bilateral vertical reflection is the only symmetry that has much prevalence in nature. While the examples provided should give some evidence of other types of symmetry in nature, we also seek to counter the claim that humans are not good at detecting other types of symmetry. 

The only study that we know of specifically dealing with wallpaper groups is \citet{clarke}. In that study, Clarke found that subjects did not do very well in a symmetry group sorting task. Further, he found that their strategy seemed to be primarily motivated by rotation. In this paper, we argue that the nature of Clarke's task was difficult for reasons other than perceiving the difference between two groups. Further, we argue that the strategy is not simply based on rotation.

\section{Research Questions}
Due to previous work diminishing the role of symmetry groups in perception, we sought to design a conclusive experiment that tested what features of symmetry were detectable to humans. Further, we wanted to compare multiple different theories of symmetry perception in order to detect which was the most probable. In order to do this, we would design a task requiring people to differentiate wallpaper groups. Our research questions can be summarized as:

\begin{enumerate}
\item {\textbf{\textit{Can people naively distinguish among the wallpaper groups?}} If the answer is yes, then we can expect that, at the very least, more features than reflection play a role in symmetry perception. Second, if people are very good at telling apart the wallpaper groups, then it's possible that symmetry, as defined by mathematics, does actually play a role in perception, fitting in nicely with a computational cognition paradigm.}
\item {\textbf{\textit{What features of symmetry drive the perception of symmetry?}} While most previous studies assumed it was reflection, specifically bilateral reflection, we are not so sure. Clarke's study claimed it was rotation; however, his experimental setup was not able to isolate different variables. If we are able to show that the best predictor of symmetry detection is rooted in the mathematical definition, it would again fit in nicely with the paradigm. Answering this question requires the comparison of multiple different models for how well they fit people's actual performance.}
\begin{enumerate}[a.]
\item {A reflection-based model. This would be in line with previous studies, and it would suggest that the primary predictor of symmetry detection is the presence or absence of reflection axes in various places. \citet{yanxibook} suggests that reflection can be decomposed into four vectors: two diagonal vectors, and two border vectors. We will use that paradigm when testing this model}
\item {A rotation-based model. This would follow Clarke's study, and assume that rotation is the primary driver of symmetry detection. To interpret this model most generously, we would look at each type of rotation for each group: for instance, if 2-fold rotation is the most important feature, then we could assume that 4-fold and 6-fold groups have it, while 3-fold groups do not.}
\item {A subset-based model. This would follow the computational cognition paradigm we are advocating, where perception uses the full set of features as defined by mathematics. To measure this, we would use the shortest distance as defined by Figure~\ref{graph}, which is a good corollary for the groups' mathematical similarity.}
\end{enumerate}
\end{enumerate}
\begin{figure}[!ht]
\centering
\includegraphics[width=0.9\columnwidth]{symper}
\caption{A screenshot of the experimental task}
\label{screenshot}
\end{figure}

\section{Experiments}
We designed an experiment where people had to distinguish between wallpaper groups. The goal of the task is to see which groups subjects can easily distinguish among and which groups are more challenging. In every trial, a subject was presented with two images from the same group and a third from one of the other sixteen groups. On the left, we have the \textit{target} image, which is the image with which they compare the other two images to. On the right, there is the \textit{goal} image, which always has the exact same group as the target image. Also, on the right is the \textit{distracter} image, which has any other wallpaper group. Figure~\ref{screenshot} shows what a single trial looks like. 

We recruited 106 participants on Mechanical Turk to participate in the experiment for compensation. Each subject performed 272 trials. This allowed each trial to have every group as the distracter image for every other group. This allows us to determine which features of the groups cause difficulty.

To generate these 272 tasks, we created every possible combination task from a small set of images from each wallpaper group. Then, each participant who accepted the HIT on Mechanical Turk was assigned a set of tasks randomly, though they received exactly one task that had any given group for the goal and the distracter.

We introduced a time limit of five seconds to ensure that the subjects were performing the task intuitively without relying on formal knowledge. This should give participants plenty of time to click the one they felt was more similar. However, when the task was given to people who have been trained in distinguishing wallpaper groups, they reported that the time limit felt too short to determine which group was which. Thus, it is likely that the results are a result of perception rather than education.

As the task can be tiring, subjects were allowed one break halfway through the experiment. Each subject was paid \$2 and a bonus for each task they successfully completed beyond chance, to encourage people to try as hard as they could on the task, instead of simply guessing. They could earn up to \$2 by answering 100\% correctly.  

Unfortunately, this incentive did not work on all participants. We excluded 10 participants from the analysis due to behavior that seemed consistent with either a lack of effort or following the directions poorly.  We used exclusion heuristics that identified participants that primarily clicked on either the top or bottom image, or if participants let a large portion of their tasks time out.


% We used three exclusion heuristics:

% \begin{enumerate}
% \item{A participant clicked the top or the bottom image more than 60\% of the time. If this was due to the system’s task generation, the probability of it occurring would be extremely low, about 0.001\%. However, 5/106 participants fell into this category, implying that their guessing strategy was slightly biased.}

% \item{A participant had more than 20\% of their tasks time out. Over-
% all, the rate of time out (exceeeding the five second time limit) was fairly low, around 4\%. Therefore, those participants with significantly higher rate of time outs either did not understand the task correctly or were not taking the task seriously. Only 4/106 participants fell into this category.}

% \item{If the system registered that a participant clicked more than 125\% of the number of tasks. Each task only required a single click to complete, so the assumption is that someone who is clicking that often is not actually looking at the images before clicking. Generally, this refers to clicks on the image. While some people have a more ``excitable'' clicking strategy, the average number of clicks was around 274, with the mode being 272. Those with significantly more were most likely clicking before the images loaded. Only 1/106 participants fell into this category.}
% \end{enumerate}


\section{Analysis}
Our two basic research questions were addressed individually.

\subsection{Question 1}
To answer the first research question, we can rely on subjects' accuracy in distinguishing the two groups. We calculated this by combining the results for each as the distracter and the target. Then, we aggregated all the subjects and took the percentage that was correct. Our hypothesis was that participants can tell apart every group from every single group. To test this, we did a simple binomial t-test comparing participants' actual performance to chance ($\pi=0.5$). This method should have a tendency towards false negatives, as we are performing $136$ tests of significance. Nonetheless, using this paradigm, we found that $135/136$ tests were significant at the $p=0.05$ level.  

A significance test naturally can have some amount of false positives and false negatives. In our study, the most difficult for participants was the distinction between “p4m” and “pmm” (see Figure~\ref{pmmp4m} for an example). Interestingly, an expert given a small amount of time should have almost no trouble distinguishing them, as the lack of 4-fold rotation is quite obvious in "pmm." Across the 96 participants, only 56.8\% of trials were successful. On a single-tailed binomial distribution, there is a p-value of $0.06$ on that number of trials. While this misses the classical $0.05$ marker for significance, every single other group by group comparison meets it. Therefore, we find it likely that with more subjects, every group would have shown itself to be statistically distinguishable. The average accuracy fell right around 76.8\%, which is clearly far better than people would perform if they were mostly guessing. Further, due to the number of significance tests, there would be some natural variation in any specific test.

Technically, the data we collected were asymmetric; that is, we have separate data from when group $A$ serves as the target and group $B$ serves as the distracter than when group $A$ serves as the distracter and group $B$ serves as the target. See Figure~\ref{fullacc} for the non-symmetric accuracies. We performed a $\chi^2$ test for symmetry to determine the reasonability of otherwise lumping the data together. This McNemar's test failed standard tests for significance $p > 0.1$; however, it was close enough $p < 0.15$ that the possibility remains of some asymmetry. Nonetheless, for the purposes of this current experiment, the data was combined.

\begin{figure}[!ht]
\centering
\includegraphics[width=0.9\columnwidth]{pmmp4m}
\label{pmmp4m}
\caption{PMM on the left (note the tiles lack reflection on the horizontal axis, but have it on the vertical axis), P4M on the right}
\end{figure}

\begin{figure}[!ht]
\centering
\includegraphics[width=0.9\columnwidth]{accuracies-grayscale}
\label{fullacc}
\caption{On the vertical we have the targets. On the top are the distracters. Main diagonal represents the aggregate accuracy in labeling that group correctly. The darker the bar, the closer to 1.0.}
\end{figure}

\subsection{Question 2}
To answer our second question, we used a Linear Mixed Effects Regression (GLMM) model. Since we wanted to predict what caused the subjects to choose correctly, we used accuracy as the response variable. Thus, the first step was to change each task to its own record. Then, each feature we considered could have a positive or negative impact on the accuracy.

Each task was boiled down to a comparison of the two symmetry group’s features. These features were whether or not the groups of the target and distracter images had the same tile shape, features determining whether the groups had the same value (Reflection, Glide Reflection, neither) along their reflection axes ($T_1, T_2, D_1, D_2$), and features for whether they had the same value (True or False) for 2-fold, 3-fold, 4-fold, and 6-fold rotation. Lastly, we also included the value for the shortest path (computed with Djikstra's algorithm) on the subgroup relation graph; for instance, the distance from “p6m” to “p6” is 1. While each of the first four refer to some basic feature of symmetry, the graph distance is included as an alternative theory. In short, it would posit that humans perceive symmetry very closely to its mathematical analysis, instead of as a collection of individual symmetries.

We used two random effect variables, which represent uncontrolled aspects of the model. The first is the actual participant who performed the task, which obviously has some effect on the accuracy. The second is the actual task they received: given that the target, goal, and distracter could each be in a different spot, there were two hundred possibilities for each of the 272 tasks. In early trials using a less random selection procedure, repeats of the same image in the same location were common enough to be noticeable by test participants. While we could have iteratively designed a uniform test set avoiding this issue, random effects allow us to control for the variance our methodology caused.

\begin{figure}[!ht]
\centering
\includegraphics[width=0.9\columnwidth]{Yanxi_Graph}
\label{graph}
\caption{The subgroup relation graph}
\end{figure}

First, we performed a side-by-side model comparison, to determine which of the models had the lowest \textit{AIC}: a metric for evaluating models based on how well they explain the data given how many variables they include. We also looked at \textit{BIC}: a metric similar to the AIC that also takes sample size into account.  As we were looking at individual records, our sample size was quite high, at roughly 26,000. However, somewhat problematically in our case, BIC uses likelihood estimation that, to some extent, tacitly assumes the model could be correct, as it compares among possible models \citep{modelselect}\citep{techreport}. In our case, our data analysis explored variables relevant to symmetry and pattern recognition, but it could not possibly capture the entirety of the variance in human perception. Thus, while we report both, we focus on AIC as the more valid metric. Lastly, we used a pseudo-R-squared metric \citep{r2}. We report the conditional, rather than marginal, r-squared due to our lack of interest in exploring our random effects. We compared a rotation-based model, a reflection-based model, and a distance-based model. Results can be found in Table~\ref{results}. As can be seen, the model based on distance was the best naive model. 

\begin{table}
\centering
\begin{tabular}{|l|ccccc|}
\hline
& Rotation & Reflection & Distance & Best & All \\ \hline
AIC & 27251.1 &  27146.5 & 27105.8 & 26857.0 & 26856.3 \\ \hline
BIC & 27308.4 & 27203.7 & 27138.5 & 26922.4 & 26962.6 \\ \hline
$r^2$& 0.1234 & 0.1293 & 0.1317 & 0.1449 & 0.1454  \\ \hline
\end{tabular}
\label{results}
\caption{Comparison of theoretically-driven models}
\end{table}

Next, we found the best overall model from the pool of features described earlier. This model included distance, but additionally included the $T_1$ axis, which would correspond to bilateral reflection symmetry, the $D_1$ axis, which is the positive diagonal, along with 4-fold and 3-fold rotation. Finally, as a control case, we compared a model including all discussed features. All of these can be found in Table~\ref{results}.
\section{Discussion}
There are several results that we obtained that require some explanation in order to sit neatly on top of the existing literature. \citet{clarke} found that people aren't very good at the task, but that the strategy they use tends to rely on rotation. However, Clarke's study relies on grouping images together. When presented with this task, it seems unlikely that most people would immediately choose to use 17 groups. Indeed, the number of sets participants made varied wildly in Clarke?s study, from 2 to 23 \citep{clarke}. This makes it somewhat difficult to analyze the actual strategy participants used. Further, if people are to choose some number of groups less than 17, rotation may be a logical method to sort them by. Importantly, Clarke's task does not isolate perception very well. It allows participants an unlimited amount of time, allowing them to use other sorts of analytical reasoning. Second, due to working memory constraints, it would be difficult for participants to recall precisely what type of image corresponded to each bin as they continued to sort. These confounds could explain the difference.

\begin{table}
\centering
\begin{tabular}{|l|cccc|}
\hline
& Estimate & Std. Error & z value & Pr(<|z|)  \\ \hline
(Intercept) & 0.95971 &  0.09950 & 9.645 & <2e-16*** \\ \hline
T1 & -0.24795 &  0.04150 & -5.975 & 2.30e-09*** \\ \hline
T2 & -0.07746 & 0.04026 & -1.924 & 0.0544 \\ \hline
D1 & -0.31416 & 0.04063 & -7.732 & 1.06e-14*** \\ \hline
D2 & 0.10197 & 0.04431 & 2.301 & 0.0214* \\ \hline
2fold & 0.01214 & 0.03540 & 0.343 & 0.7316 \\ \hline
3fold & -0.18056 & 0.04197 & -4.302 & 1.69e-05*** \\ \hline
4fold & 0.32353 & 0.04282 & 7.555 & 4.19e-14*** \\ \hline
6fold & -0.07538 & 0.04484 & -1.711 & 0.0870 \\ \hline
tile & -0.07891 & 0.05556 & -1.420 & 0.1555 \\ \hline
distance & 0.22420 & 0.01872 & 11.978 & <2e-16*** \\ \hline
\end{tabular}
\label{fixeff}
\caption{Fixed Effects}
\end{table}

In general, this paper advocates for a model of symmetry/pattern detection that is mathematically driven. We believe the subgroup distance to be one method for capturing the method humans perceive regularities. Importantly, it is not the only one and it may not even be the best one. By using Dijkstra's algorithm, we weighted each edge to be exactly equal. However, that's not necessarily the case. While "p1" and "p2" differ by exactly one symmetry (the 2-fold rotation), "p2" and "pgg" differ by three symmetries. Another possible view is to calculate the distance using the Hamming distance between each pair, rather than the relation graph. However, there are clearly multiple definitions of even the same mathematical model when it is operationalized for perception, which complicates the story.

As the model containing only subgroup distance outperformed the other simple models, we find it likely that subgroup distance is the single most useful feature. However, it should be clear from our results displayed earlier that distance alone does not tell the whole story. Thus, we interpret the additional features in the converged model as having an overly important role in the process of pattern recognition. In other words, human perception might use some quick heuristics involving these symmetries instead of doing a full analysis. 

\begin{table}
\centering
\resizebox{\columnwidth}{!}{%
\begin{tabular}{|l|cccccccccc|}
\hline
(Intercept) & T1 & T2 & D1 & D2 & 2fold & 3fold & 4fold & 6fold & tile & distance  \\ \hline
T1 & -0.191 &   & & & & & & & & \\ \hline
T2 & -0.090 & -0.556 &  & & & & & & &  \\ \hline
D1 & -0.216 & 0.111 & 0.040 & & & & & & & \\ \hline
D2 & -0.051 & -0.005 & -0.137 & -0.572 & & & & & & \\ \hline
2fold & -0.472 & 0.113 & -0.010 & 0.136 & -0.061 & & & & & \\ \hline
3fold & -0.127 & 0.148 & -0.090 & 0.088 & 0.039 & -0.068 & & & & \\ \hline
4fold & -0.464 & 0.179 & -0.121 & 0.172 & -0.245 & 0.160 & 0.237 & & &  \\ \hline
6fold & -0.245 & 0.001 & 0.048 & 0.055 & -0.192 & 0.060 & -0.413 & 0.029 & & \\ \hline
tile & -0.169 & -0.210 & 0.241 & -0.141 & 0.072 & 0.180 & -0.525 & -0.292 & 0.189 & \\ \hline
distance & -0.713 & 0.052 & 0.082 & 0.046 & 0.116 & 0.405 & -0.029 & 0.296 & -0.026 & 0.350 \\ \hline
\end{tabular}%
}
\label{corr}
\caption{Correlation of Fixed Effects (note that this is the empirical correlation rather than the analytical relationship)}
\end{table}

For instance, why are $T_1$ and $D_1$ significant (the vertical and positive diagonal axis), but not $T_2$ and $D_2$ (the horizontal and negative diagonal axis)? While it seems easy to attribute it to perception, it is important to realize there are substantial correlations among features in the wallpaper groups. In some cases, this should be obvious: if a wallpaper group has 6-fold rotation, it is guaranteed to have both 2-fold rotation and 3-fold rotation by the definition. In the case of the reflection axes, there are substantial correlations (about $0.5$) between $D_1$ and $D_2$ and between $T_1$ and $T_2$. Thus, while it's somewhat difficult to disentangle their independent effects, it seems clear that both $T_1$ and $D_1$ are at least more important, as the models including them instead of $T_2$ and $D_2$ perform better on the earlier discussed metrics. However, $D_2$ is also a significant predictor even considering $D_1$, while $T_2$ is not a significant predictor while considering $T_1$. A possible explanation for this could result in faces, with $D_1$ and $D_2$ representing faces at an angle. As mentioned earlier, there are some biological reasons for believing humans are adept at recognizing faces \cite{ffa}.     

The overabundance of the role of reflection symmetry in the literature could be due to the lack of knowledge of researchers of other types of symmetry. Further, it could be related to vast literature on faces, which contains reflection symmetry but lacks all other types. While our results show that people obviously can detect reflection symmetry, it does not tell the whole story.

In the case of rotation, the most significant features were 3-fold and 4-fold. Interestingly, the effect of 4-fold rotation is positive (if both groups both have 4-fold rotation or both don't, accuracy improves), while the effect of 3-fold rotation is negative. This means that groups with 4-fold rotation are not necessarily easy to determine from groups without it, but they are easy to tell apart internally. This is very possibly due to 4-fold rotation's correlation with the other features.

For 3-fold rotation, it is possible that (unlike 4-fold rotation), it plays a more significant role due to its lack of ubiquity. Many objects humans interact with in the modern world contain 4-fold rotation, to the point where perhaps it is so common that it is no longer a useful heuristic. On the other hand, 3-fold rotation is a bit more rare, and thus may have some special significance.

Lastly, the highly significant effect of distance is interesting. The simplest reading of this is that people perceive group to group similarity as more than any individual feature. Computer vision has long been inspired by wallpaper groups \citep{yanxi1}\citep{yanxi2}, and this result lends further credence to the idea that they are some fundamental part of human perception.
\section{Conclusion}
This paper gives an overview and preliminary results to expand work on perception of symmetry. Presently, very few researchers have done any experiments on types of symmetry other than bilateral reflection; however, we present evidence that humans can and do perceive the other types of symmetry. We found that all but one pair of wallpaper groups are distinguishable (p < 0.5) and all are likely distinguishable (p < 0.1). After comparing models, we found that certain axes in reflection and certain types of rotation both play a role in symmetry detection. Notably, we also found a highly significant effect in the subgroup distance.
\bibliographystyle{apacite}

\setlength{\bibleftmargin}{.125in}
\setlength{\bibindent}{-\bibleftmargin}

\bibliography{symmetry,symmetry2,perception,yanxibook,statistics,statistics2,statistics3}


\end{document}
