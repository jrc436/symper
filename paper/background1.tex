As \textit{symmetry} can be thought of as a transformation on an object that leaves that object exactly the same. This could apply to a variety of objects: for instance, if someone has an infinitely long loop of musical notes, shifting all of the notes forward in time by the length of the loop would not change the music at all. However, this paper focuses on symmetry as it relates to two dimensional images.

Importantly, the way humans detect symmetry is not necessarily reliant on perfect symmetry. While textures which have been manipulated can pose a serious problem for computer vision algorithms attempting to find patterns, humans are quite good at it \cite{nearregular}.

In any type of symmetry, we can imagine the concept of a symmetry \textit{group} or set. In this context, the group refers to the collection of objects with the same set of symmetries as other objects in the collection. For instance, faces could be considered a group whose only symmetry is a reflection axis down the center.

The class of two dimensional images that are infinitely repeating are known as wallpapers. There are exactly seventeen wallpaper groups, as has been well-known for over a hundred years \cite{wallpaper-proof}. In other words, every two dimensional repeating image has one of seventeen different sets of symmetries.