\section{Introduction}
The concept of \emph{symmetry} refers to a common property of everyday objects and images. 
Symmetry is present when, for instance, one half of the object appears mirrored along an axis. Faces are a good example: they are approximately symmetrical along a vertical axis. Symmetry has received much attention as a feature of visual perception and attention (see \cite{review} for a review). However, symmetry as a geometric concept is much more general: symmetry refers to any transformation of an object or image that leaves it with the exact same appearance. Mathematically, exactly four types of symmetry are distinguished in two dimensional images.

Prior work has focused largely on reflection, such as the example of faces. It has even been suggested that this is the only type of symmetry humans can process at once \citet{bio}. In this study, we present empirical data suggesting that this is not the case. The experiments are designed to test whether people perceive differences in symmetry, and whether they can identify and distinguish all groups of objects that are formed by two dimensional symmetries. 

The vernacular term symmetry generally refers to \emph{reflection}, with many studies focused on bilateral reflection. Bilateral reflection describes situations when a single vertical axis separates two mirrored halves. Reflection is one of the four types of two dimensional symmetry; the others are rotation, glide reflection, and translation. Generally, symmetry can be thought of as a transformation on any object that leaves that object exactly the same. This can apply to a variety of objects beyond even images. For instance, if we consider an infinitely long loop of musical notes, shifting all the notes forward in time by the length of the loop would not change the music at all. However, this paper focuses on symmetry as it relates to two dimensional images.

We will use \emph{wallpapers} to create images that use combinations of these symmetry types. Wallpapers are two-dimensional images made from an infinitely repeating symmetric \textit{tile}. These tiles are characterized by a specific set of symmetries. There are exactly seventeen wallpaper groups, as has been well-known for over a hundred years \citet{wallpaper-proof}. In other words, every two-dimensional repeating image has one of seventeen sets of symmetries. All wallpapers can be arranged in a hierarchy. 

Through statistical modeling of our empirical data, we will argue that this mathematically well-motivated hierarchy of symmetry groups actually has relevance in perceptual processing. 