\section{Introduction}
Symmetry has oft been studied as a feature of visual perception and attention (see \citet{review} for a review). However,  the cognitive science community has focused primarily on bilateral reflection symmetry. This study will give an overview of current work on the field before moving onto the features of symmetry that have been largely ignored. 

Symmetry refers to a transformation of an object or image that leaves it with the exact same appearance. There are four primitive types of symmetry: rotation, reflection, glide reflection, and translation. The vernacular term symmetry generally refers to reflection, with the majority of studies focused on bilateral reflection, where the axis is in the center of the object or image.

This paper will focus on \textit{wallpapers}: a type of infinitely repeating symmetric image that is characterized by a specific set of symmetries. First, we will provide the reader with the necessary background to understand this conceptualization, then we will review past work on symmetry, and then we will present a set of findings that show humans are quite adept at detecting other types of symmetry besides bilateral reflection.



