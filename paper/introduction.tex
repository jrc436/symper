\section{Introduction}
While symmetry has oft been studied as a feature of visual perception, the cognitive science community has focused primarily on bilateral reflection symmetry. However, symmetry is a well-defined concept beyond finding the center of a lion's face. This paper seeks to show the usefulness of such concepts to a community interested in studying visual perception.

Symmetry, in terms of images, refers to some basic transformation being performed to the image, but the image staying identical. There are four primitive types of symmetry: rotation, reflection, glide reflection, and translation. For instance, a hexagon has 6-fold rotation symmetry, because it can be rotated on six points and remain the same. Reflection refers for half of the image to be mapped onto the other half. Notably departing from previous studies, this type of reflection does not need to be vertical. While faces only have one reflection axis, a square has four, and a circle has an infinite number.

Glide reflection and translation are a departure from what most people think of when they think of symmetry. Glide reflection refers to a skewed reflection, where the objects reflect, but they're also moved. For instance, footsteps is a simple example of glide reflection.

Translation symmetry refers to an object that repeats. In a short, informal scale, this could be thought of as a fence, where the \textit{tile}, the repeating part, would be individual fence posts. In a mathematical sense, for an object to have translational symmetry, it must be infinitely large: otherwise, after moving each individual fence post, the position of the fence would have moved.

In two dimensions, an infinitely large pattern that has translation symmetry is referred to as a \textit{wallpaper}. While all wallpapers have translational symmetry, they can still be divided based on their other symmetries. Interestingly, there are actually only seventeen real combinations of the set of symmetries \cite{wallpaper_proof}. Intuitively, this is because each tile which, must contain all of the other presented symmetries, must also then fit together with an infinite number of tiles of the exact same type. 

Despite the formal mathematical work having existed for 

only seventeen types of these wallpapers. However, little re-
search has been done showing the relationship between these

groups and human perception.

This study is motivated in part by the only prior work we

%could find on the subject (Clarke, Green, Halley, & Chantler,

2011). In that study, Clarke found that subjects did not do

very well in a symmetry group sorting task. Further, he found

that their strategy seemed to be primarily motivated by rota-
tion. In this paper, we argue that the nature of Clarke’s task

was difficult for reasons other than perceiving the difference

between two groups. Further, we argue that the strategy is not

simply based on rotation.