We recruited 106 subjects from the Amazon Mechanical Turk platform to compare among the wallpaper groups. Their basic task was to match two wallpapers from the same group. The third image could be from any of the other sixteen groups. In this way, we could distinguish how difficult it is for humans to tell the groups apart. Further, we only allowed each participant five seconds, in order to force them to rely on perceptual, rather than analytical, processes. All subjects were compensated for their time, though 12 were excluded for behavior consistent with not taking the task seriously. \linebreak 

We used simple binomial t-tests to determine how accurate humans were at distinguishing among every wallpaper group. We also used a Linear Mixed Effects Model to determine what drives the variance in distinguishing between two pairs of wallpapers. Our features in the model consisted of the symmetries as defined by the group-theoretic analysis of the wallpaper groups. We additionally considered \textit{subgroup distance}, an aggregate measure referring to the shortest path in the subgroup graph, and \textit{tile shape}, the piece of the image that is infinitely translated.