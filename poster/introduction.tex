\begin{block}{Introduction}
The concept of \emph{symmetry} refers to a common property of everyday objects and images. Symmetry is present when, for instance, one half of the object appears mirrored along an axis. Faces are a good example: they are approximately symmetrical along a vertical axis. Symmetry has received much attention as a feature of visual perception and attention (see \citep{review} for a review). However, symmetry as a geometric concept is much more general: symmetry refers to any transformation of an object or image that leaves it with the exact same appearance. Mathematically, exactly four types of symmetry are distinguished in two dimensional images.\linebreak

Prior work has focused largely on reflection. It has even been suggested that this is the only type of symmetry humans can process at once \citep{bio}. In this study, we present empirical data suggesting that this is not the case. The experiments are designed to test whether people perceive differences in symmetry, and whether they can identify and distinguish all groups of objects that are formed by two dimensional symmetries. \linebreak

We will use \emph{wallpapers} to create images that use combinations of these symmetry types. Wallpapers are two-dimensional images made from an infinitely repeating symmetric \textit{tile}. These tiles are characterized by a specific set of symmetries. There are exactly seventeen wallpaper groups, as has been well-known for over a hundred years \citep{wallpaper-proof}. In other words, every two-dimensional repeating image has one of seventeen sets of symmetries. These wallpapers can be arranged in a hierarchy, which we argue relates to how they are perceived.
\end{block}